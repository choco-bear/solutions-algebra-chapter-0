\newpage
\begin{center}
\fbox{
\parbox{13cm}{
Copyright (C) 2023 doyoung @ \href{https://github.com/choco-bear/}{https://github.com/choco-bear/}. \\
Permission is granted to copy, distribute and/or modify this
document under the terms of the GNU Free Documentation License,
Version 3.0 or any later version published by the Free Software
Foundation; with the Invariant Sections being just "GNU
Manifesto" and just "Prologue", with no Front-Cover Texts, and with no Back-Cover
Texts.  A copy of the license is included in the section
entitled "GNU Free Documentation License".
}}

%\vspace{48pt}
%\emph{Special thanks to the following contributers:}
\end{center}

\chapter*{Prologue}
\addcontentsline{toc}{chapter}{Prologue}

\section*{Introduction}
\addcontentsline{toc}{section}{Introduction}
\markboth{\MakeUppercase{Prologue}}{\MakeUppercase{Introduction}}

``Algebra: Chapter 0." Even the title of this textbook suggests a beginning, a starting point, a foundation from which to explore the complex and captivating world of algebra. It's a signal to the reader that this is where the journey begins, an invitation to set out on a path filled with intriguing problems, elegant solutions, and a deeper understanding of the mathematical principles that govern our world.

As an undergraduate student, I found myself drawn to this beginning, to ``Chapter 0," and to the idea of delving into algebra from the ground up. My curiosity was piqued, but I also recognized that curiosity alone might not be enough to navigate the intricacies of algebra. The challenges could be overwhelming, the concepts elusive, and the solutions often just out of reach.

That realization led me to embark on a personal journey of self-study and exploration, and this solution manual is the result.

In these pages, I have documented my path through ``Algebra: Chapter 0," providing solutions, insights, and explanations along the way. This manual is not the work of an expert; it's the work of a fellow traveler, one who has grappled with the same questions, puzzled over the same problems, and rejoiced in the same moments of understanding.

My goal is to make this journey more accessible to others who are drawn to the world of algebra. Whether you're a fellow student, a teacher, or simply someone with a curiosity about mathematics, I hope you'll find value in these pages. Each solution is presented with clarity and care, reflecting not only the method to arrive at the answer but the thought process and insights that led me there.

So, here's to ``Chapter 0," to beginnings, and to the joy of learning. Join me on this adventure, and let's explore the fascinating landscape of algebra together. Let's embrace the challenges, celebrate the discoveries, and take pleasure in the knowledge that we are part of a community of learners, all striving to understand, all starting from the same point: Chapter 0.

Welcome to the journey.

\newpage
\section*{Some important points}
\addcontentsline{toc}{section}{Some important points}
\markboth{\MakeUppercase{Prologue}}{\MakeUppercase{Some important points}}
There are a few important points to note here:
\begin{itemize}
    \item The solution is \emph{only} hosted on my GitHub page
        \begin{center}
		  \href{https://github.com/choco-bear/algebra-chapter-0-solutions}{https://github.com/choco-bear/algebra-chapter-0-solutions}.
        \end{center} 
        If you find this document outside this page, you might have an outdated version of the solution which might have errors, so please be aware.
    \item I will update the solution irregularly.
    \item I've tried to reflect errata \href{https://www.math.fsu.edu/~aluffi/algebraerrata.2009/Errata.html}{https://www.math.fsu.edu/~aluffi/algebraerrata.2009/Errata.html} as much as possible.
    \item If you found an error in the solutions, typos, bad grammar or want to give an advise on LaTeX formatting, etc., don't hesitate to open an issue or a pull request on my repo. 
\end{itemize}

Best,

\begin{flushright}
doyoung @ \href{https://github.com/choco-bear/}{https://github.com/choco-bear/} \\
Department of Computer Science \& Engineering, Seoul National University \\
Updated \specialdate\today \\
v1.4.5
\end{flushright}