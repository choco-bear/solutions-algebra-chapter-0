%%             theoremstyle                %%
\newtheoremstyle{break}%    name
  {}%                       Space above, empty = `usual value'
  {2em}%                    Space below
  {\upshape\setlength{\parskip}{0pt}\setlength{\parindent}{1em}}%               Body font
  {}%                       Indent amount (empty = no indent, \parindent = para indent)
  {\bfseries}%              Thm head font
  {.}%                      Punctuation after thm head
  {\newline}%               Space after thm head: \newline = linebreak
  {}%                       Thm head spec

\theoremstyle{break}

%%                theorems                 %%
\newtheorem{theorem}{Theorem}[subsection]
\renewcommand*{\thetheorem}{
\ifnum0=\value{chapter}\relax
    \ifnum0=\value{section}\relax
        % If chapter == 0 and section == 0, then 1. 2. 3. ...
    \else
        \thesection.
        % If chapter == 0 and section != 0, then section.1. section.2. section.3. ...
    \fi
\else\ifnum0=\value{section}\relax
    \thechapter.
    % If section == 0, then chapter.1. chapter.2. chapter.3. ...
\else\ifnum0=\value{subsection}\relax
    \thesection.
    % If section != 0 and subsection == 0, then chapter.section.1. chapter.section.2. chapter.section3. ...
\else
    \thesubsection.
    % If subsection != 0, then chapter.section.subsection.1. chapter.section.subsection.2. chapter.section.subsection.3. ...
\fi\fi\fi\arabic{theorem}\!\!
}
\newtheorem{corollary}{Corollary}[theorem]
\newtheorem{remark}[theorem]{Remark}
\newtheorem{proposition}[theorem]{Proposition}
\newtheorem{lemma}[theorem]{Lemma}
\newtheorem{definition}{Definition}
\newtheorem{example}{Example}

%%                exercises                %%
\newtheorem{exercise}{Exercise}[subsection]
\renewcommand*{\theexercise}{
\ifnum0=\value{chapter}\relax
    \ifnum0=\value{section}\relax
        % If chapter == 0 and section == 0, then 1. 2. 3. ...
    \else
        \thesection
        % If chapter == 0 and section != 0, then section.1. section.2. section.3. ...
    \fi
\else\ifnum0=\value{section}\relax
    \thechapter
    % If section == 0, then chapter.1. chapter.2. chapter.3. ...
\else\ifnum0=\value{subsection}\relax
    \thesection.\hspace{-0.4em}
    % If section != 0 and subsection == 0, then chapter.section.1. chapter.section.2. chapter.section3. ...
\else
    \thesubsection.\hspace{-0.4em}
    % If subsection != 0, then chapter.section.subsection.1. chapter.section.subsection.2. chapter.section.subsection.3. ...
\fi\fi\fi\arabic{exercise}\!\!
}

\makeatletter 
\renewenvironment{proof}[1][\proofname]{
    \par \pushQED{\qed}% 
    \normalfont\topsep6\p@\@plus6\p@\relax 
    \trivlist 
    \item\relax {
        \itshape #1\@addpunct{.}
    }\hspace\labelsep\ignorespaces \setlength{\parindent}{0pt}\par
}{% 
    \popQED\endtrivlist\@endpefalse 
} 
\makeatother
\newenvironment{solution}{
    \begin{proof}[Solution]
}{
    \end{proof}\vspace{5mm}
}

\newcounter{problem}[section] \setcounter{problem}{0}
\newenvironment{_exercise}{
    \refstepcounter{problem} \label{prob:\Roman{chapter}.\arabic{section}.\arabic{problem}} \begin{exercise}
}{
    \end{exercise}
}
\newcommand{\problem}[2]{\noindent\begin{minipage}{\textwidth}\begin{_exercise}\label{exer:\Roman{chapter}.\arabic{section}.\arabic{exercise}}#1\end{_exercise}\begin{solution}#2 \end{solution}\end{minipage}}
\newcommand{\problembr}[2]{\begin{samepage}\begin{_exercise}\label{exer:\Roman{chapter}.\arabic{section}.\arabic{exercise}}#1\end{_exercise}\begin{solution}#2 \end{solution}\end{samepage}}

\labelformat{exercise}{Exercise \Roman{chapter}.\arabic{section}.\arabic{exercise}}
\labelformat{problem}{Exercise \arabic{section}.\arabic{problem}}

%%                qed symbol               %%
\renewcommand\qedsymbol{$\blacksquare$}